
% -----------------------------------------------------------------------------------------------------------------
\pagenumbering{gobble}% Remove page numbers (and reset to 1)
\clearpage
\thispagestyle{empty}

{\fontsize{14pt}{1em}\selectfont \textbf{Resumen}}

En este trabajo hemos pretendido y conseguido diseñar nuevos métodos iterativos para la resolución de ecuaciones y sistemas de ecuaciones no lineales; éstos son, comparados con los ya existentes, muy eficientes y estables. Posteriormente hemos aplicado dichos métodos a algunas ecuaciones no lineales que tienen un interés físico reconocido: el problema de Bratu y la ecuación de Burgers; en ambos casos el objetivo es encontrar la solución de una ecuación en derivadas parciales no lineal. Dado que el diseño ha resultado en famílias de métodos en vez de métodos únicos, hemos utilizado las técnicas dinámicas para elegir qué miembros de esa familia (a pesar de que todos tienen el mismo orden de convergencia) son las más estables. Además, también hemos diseñado y estudiado una nueva forma de discretizar la ecuación de Burgers con el objetivo de aumentar la precisión de la solución y simplificar el proceso de la obtención de ésta.\\\\

{\fontsize{14pt}{1em}\selectfont \textbf{Resum}}

En este treball hem pretés i aconseguit dissenyar nous mètodes iteratius per a la resolució d'equacions i sistemes d'equacions no lineals; estos són, comparats amb els ja existents, molt eficients i estables. Posteriorment hem aplicat els dits mètodes a algunes equacions no lineals que tenen un interés físic reconegut: el problema de Bratu i l'equació de Burgers; en ambdós casos l'objectiu és trobar la solució d'una equació en derivades parcials no lineal. Atés que el disseny ha resultat en famílies de mètodes en compte de mètodes únics, hem utilitzat les tècniques dinàmiques per a triar quins membres d'eixa família (a pesar que tots tenen el mateix orde de convergència) són les més estables. A més, també hem dissenyat i estudiat una nova forma de discretizar l'equació de Burgers amb l'objectiu d'augmentar la precisió de la solució i simplificar el procés de l'obtenció d'esta.\\\\

{\fontsize{14pt}{1em}\selectfont \textbf{Abstract}}

In this work we have tried and succeeded in designing new iterative methods for solving nonlinear equations and systems; these are compared with the already existing ones and it has been found that they are highly efficient and stable. Then, we applied these methods to some nonlinear equations which have a recognized physical interest: Bratu's problem and Burgers's equation; in both cases the goal is to find the solution of a nonlinear partial differential equation. Because the design has resulted in families of methods instead of unique methods, we used the dynamical techniques in order to choose which members of the family (although all of them have the same order of convergence) are the most stable. Furthermore, we have also designed and studied a new way of discretizing Burgers's equation in order to increase the accuracy of the solution and simplify the obtaining process of it.
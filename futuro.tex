
% -----------------------------------------------------------------------------------------------------------------
%\newpage
%\thispagestyle{empty}
%\rule{\linewidth}{2pt}
\chapter{Líneas de investigación futuras}

Respecto al proceso de discretización de una ecuación en derivadas parciales, existen dos conceptos que determinan el buen funcionamiento de este: la consistencia y la estabilidad. La consistencia estriba en demostrar analíticamente lo que la intuición nos sugiere, eso es, que a medida que cogemos un paso espacial $h$ más pequeño, la aproximación de la ecuación mejora. Cuando eso no se cumple decimos que el método no es consistente. La estabilidad puede entenderse como el conjunto de condiciones que el proceso de discretización (elección de paso espacial $h$ y temporal $k$) debe cumplir para que el método funcione correctamente. Mientras que para los métodos explícitos sí suelen existir dichas condiciones, para los métodos implícitos no es así (por lo menos no en el caso de ecuaciones lineales). Así pues, aunque el método de Crank-Nicholson (tipo particular de método implícito) está perfectamente estudiado para ecuaciones en derivadas parciales lineales, es decir, su consistencia y estabilidad ha sido demostrada, este no lo está para el caso de ecuaciones no lineales. En el Capítulo \ref{articuloburger}, donde hemos aplicado dicho método de Crank-Nicholson sobre una ecuación en derivadas parciales no lineal (Burgers), hemos asumido que el método era consistente y que carecía de condiciones a cumplir, ya que éste ha funcionado para todas las pruebas numéricas realizadas, sin embargo queda como línea de investigación futura demostrar la consistencia y la estabilidad del método analíticamente, cuyo proceso es complejo.

De la misma forma que en la Sección \ref{seccionlichen} se ha presentado una familia de métodos óptimos, de dos pasos, con orden de convergencia 4, cuya aceleración de la convergencia residía en el uso de una función peso aplicada sobre el corrector, queda como línea de investigación futura el estudiar si esta misma idea puede aplicarse también a la familia de métodos de orden 3, presentada en la Sección \ref{seccionHG}. Así pues, el objetivo sería aumentar su orden en una unidad, y de esta forma convertirlo en óptimo (bajo la conjetura de Kung-Traub), para su uso en ecuaciones no lineales.

Por simplicidad, en la Sección \ref{seccionHGsistemas} hemos aplicado y estudiado el método de Homeier generalizado sobre el caso unidimensional del problema de Bratu, sin embargo dicho problema también puede ser considerado como un caso multidimensional. Así pues, queda como línea de investigación futura el extender el estudio del comportamiento del método presentado en dicha Sección \ref{seccionHGsistemas} al caso multidimensional del problema de Bratu.

En el Capítulo \ref{articuloburger} hemos demostrado la estabilidad del método de Crank-Nicholson mediante la aplicación del método de Newton, Traub y M5 sobre la ecuación de Burgers. El hecho de haber usado estos tres métodos en vez de los presentados en las secciones \ref{seccionlichensistemas} y \ref{seccionHGsistemas}, es debido a que en el momento de la redacción de dicho Capítulo \ref{articuloburger}, dichos métodos todavía no existían. Así pues, queda como línea de investigación futura el comprobar el comportamiento del método de Crank-Nicholson también mediante la aplicación sobre la ecuación de Burgers de los métodos diseñados en las secciones \ref{seccionlichensistemas} y \ref{seccionHGsistemas}.
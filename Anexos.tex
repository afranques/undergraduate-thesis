
% -----------------------------------------------------------------------------------------------------------------
%\newpage
%\thispagestyle{empty}
%\rule{\linewidth}{2pt}
%\chapter{Anexos}

\appendix

\chapter{Análisis dinámico}

\section{Representación del plano de parámetros de un método iterativo}
\begin{lstlisting}
	function [I,c]=planoParbiparam(axini,axfin,ayini,ayfin,puntos,maxiter)
		puntos=puntos+1;
		ax=linspace(axini,axfin,puntos);
		ay=linspace(ayini,ayfin,puntos);
		[AX,AY]=meshgrid(ax,ay);
		A=complex(AX,AY);
		I=zeros(puntos);
		R=zeros(puntos);
		G=zeros(puntos);
		B=zeros(puntos);
		c=zeros(puntos);
		
		for j=1:puntos
			for k=1:puntos
				it=0;
				aa=A(j,k);
				c1=(1-4*aa+6*aa^2+sqrt(1-8*aa+19*aa^2-12*aa^3))/(3*aa-6*aa^2);
				raizencontrada=0;
				while it<maxiter && raizencontrada==0
					c1=(c1^3*(-1 + aa*(2 + c1)))/(aa - c1 + 2*aa*c1);
					it=it+1;
					if abs(c1)<1e-3 || abs(c1)>1000
						R(j,k)=1-it/maxiter*10;
						raizencontrada=1;
					end
				end
			end
		end
		
		I(:,:,1)=(R(:,:));
		I(:,:,2)=(G(:,:));
		I(:,:,3)=(B(:,:));
		
		imshow(I,'Xdata',[axini axfin], 'Ydata', [ayini ayfin])
		xlabel('Re\{\beta\}')
		ylabel('Im\{\beta\}')
	end
\end{lstlisting}

\section{Representación del plano dinámico de un método iterativo}
\begin{lstlisting}
	function [I,it]=din_familia_param_pol2(gamma,limites,puntos,maxiter)
		xini=limites(1); xfin=limites(2); yini=limites(3); yfin=limites(4);
		syms x 
		aa=gamma;
		Op=(x^3*(-1 + aa*(2 + x)))/(aa - x + 2*aa*x);
		fOp=matlabFunction(Op);
		Opx=simplify(Op-x);
		pf=double(solve(Opx)),%,pause,
		dOp=factor(diff(Op));
		pc=double(solve(dOp));
		adOp=matlabFunction(abs(dOp));
		inda=double(abs(adOp(pf)))<1;
		pa=pf(inda==1);
		nr=length(pa);
		
		% Hacemos que la imagen tenga un numero impar de puntos
		if(mod(puntos,2)==0)
			puntos=puntos+1;
		end
		
		% Creamos la malla de puntos complejos sobre la que trabajaremos
		dx=xfin-xini; dy=yfin-yini; d=max(dx,dy);
		paso=d/puntos;
		x=xini:paso:xfin;
		y=yini:paso:yfin;
		[X,Y]=meshgrid(x,y);
		z=complex(X,Y);
		
		% Inicializacion de las matrices
		it=zeros(size(z));
		r1=zeros(size(z)); r2=zeros(size(z)); r3=zeros(size(z));
		R=zeros(size(z)); G=zeros(size(z)); B=zeros(size(z));
		[f,col]=size(z);
		for j=1:f
			for k=1:col
				s=z(j,k); raizencontrada=0;
				while (raizencontrada==0 && it(j,k)<maxiter)
					s=fOp(s);
					it(j,k)=it(j,k)+1;
					if norm(double([real(s)-real(pa(1)) imag(s)-imag(pa(1))]))<1e-3
						r1(j,k)=maxiter-1.5*it(j,k);
						R(j,k)=r1(j,k)/maxiter;
						G(j,k)=r1(j,k)/maxiter*102/255;
						raizencontrada=1;
					else if (length(pa)>=1&&norm(double(s))>.8*1e+3) || isnan(abs(s))==1 
					|| isinf(s)==1
						r2(j,k)=maxiter-it(j,k);
						R(j,k)=r2(j,k)/maxiter*40/255;
						G(j,k)=r2(j,k)/maxiter*80/255;
						B(j,k)=r2(j,k)/maxiter;
						raizencontrada=1;
					else if length(pa)>1&&norm(double([real(s)-real(pa(2))
					imag(s)-imag(pa(2))]))<1e-3
						r3(j,k)=maxiter-1.5*it(j,k);
						R(j,k)=r3(j,k)/maxiter*41/255;
						G(j,k)=r3(j,k)/maxiter*230/255;
						B(j,k)=r3(j,k)/maxiter*26/255;
						raizencontrada=1;
					else if  length(pa)>2&&norm(double([real(s)-real(pa(3))
					imag(s)-imag(pa(3))]))<1e-3
						r4(j,k)=maxiter-1.5*it(j,k);
						R(j,k)=r4(j,k)/maxiter*230/255;
						G(j,k)=r4(j,k)/maxiter*41/255;
						B(j,k)=r4(j,k)/maxiter*26/255;
						raizencontrada=1;
					end end end end
				end
			end
		end
		I(:,:,1)=R(:,:);
		I(:,:,2)=G(:,:);
		I(:,:,3)=B(:,:);
		imshow(I,'Xdata',[xini xfin], 'Ydata', [yini yfin])
		xlabel('IRe\{z\}'); ylabel('IIm\{z\}');
	end
\end{lstlisting}

\chapter{Métodos iterativos}

\section{Método de Newton}
\begin{lstlisting}
	incr=tol+1;
	incramp=incr;
	iter=0;
	while (incr+incramp)>tol & iter<maxiter
		delta=-dF\F;
		x0=x0+[0,delta',0];
		incr=norm(double(delta));
		incramp=norm(double(F));
		iter=iter+1;
	end
\end{lstlisting}

\section{Método de Traub}
\begin{lstlisting}
	incr=tol+1;
	incramp=incr;
	iter=0;
	while (incr+incramp)>tol & iter<maxiter
		delta2=-dFx\Fx;
		y0=x0+[0,delta2',0];
		delta=-dFx\(Fx+Fy);
		x0=x0+[0,delta',0];
		incr=norm(double(delta));
		incramp=norm(double(Fx));
		iter=iter+1;
	end
\end{lstlisting}

\section{Método M5 (usado en la Sección \ref{articuloburger})}
\begin{lstlisting}
	incr=tol+1;
	incramp=incr;
	iter=0;
	while (incr+incramp)>tol & iter<maxiter
		delta3=-dFx\Fx;
		y0=x0+[0,delta3',0];
		delta2=-dFx\(Fx+5*Fy);
		z0=x0+[0,delta2',0];
		delta=-dFx\(Fx+5*Fy+0.2*(-16*Fy+Fz));
		x0=x0+[0,delta',0];
		incr=norm(double(delta));
		incramp=norm(double(Fx));
		iter=iter+1;
	end
\end{lstlisting}

\section{Método M4 (usado en la Sección \ref{seccionlichensistemas})}
\begin{lstlisting}
	incr=tol+1;
	incramp=incr;
	iter=0;
	while (incr+incramp)>tol & iter<maxiter
		delta2=-(2/3)*dFx\Fx;
		y0=x0+[0,delta2',0];
		matrizu=dFx\(dFy-dFx);
		I=eye(n-1);
		delta=-(I-(3/4)*(I+(3/2+param)*matrizu)\(matrizu+param*matrizu*matrizu))*(dFx\Fx);
		x0=x0+[0,delta',0];
		incr=norm(double(delta));
		incramp=norm(double(Fx));
		iter=iter+1;
	end
\end{lstlisting}
Donde $param$ es el valor del parámetro de la familia, es decir el miembro en concreto seleccionado.

\section{Método HG}
\begin{lstlisting}
	incr=tol+1;
	incramp=incr;
	iter=0;
	while (incr+incramp)>tol & iter<maxiter
		delta2=-aa*dFx\Fx;
		y0=x0+[0,delta2',0];
		delta=-(bb*dFy\Fx+cc*dFx\Fx);
		x0=x0+[0,delta',0];
		incr=norm(double(delta));
		incramp=norm(double(Fx));
		iter=iter+1;
	end
\end{lstlisting}
Donde $aa$ es el valor del parámetro de la familia, es decir el miembro en concreto seleccionado.
\chapter{Diseño de nuevos métodos iterativos mediante Wolfram Mathematica}
A continuación se muestra el conjunto de instrucciones usadas en Wolfram Mathematica para la obtención del método HG, presentado en la Sección \ref{seccionHG}.

\textbf{Fx = dFa SeriesData[e, 0, \{0, 1,  C2, C3, C4\}, 0, 5, 1]}\\
$\text{dFa} e+\text{C2} \text{dFa} e^2+\text{C3} \text{dFa} e^3+\text{C4} \text{dFa} e^4+O[e]^5$

\textbf{dFx = D[Fx, e]}\\
$\text{dFa}+2 \text{C2} \text{dFa} e+3 \text{C3} \text{dFa} e^2+4 \text{C4} \text{dFa} e^3+O[e]^4$

\textbf{g = Simplify[e - a*Fx/dFx]}\\
$(1-a) e+a e^4 \left(4 \text{C2}^3-7 \text{C2} \text{C3}+3 \text{C4}\right)-2 e^3 \left(a \left(\text{C2}^2-\text{C3}\right)\right)+a \text{C2} e^2+O[e]^5$

\textbf{dFy = Simplify[dFx /. e -> g]}\\
$\text{dFa}$+$e^3 \left(-4 a \text{C2} \text{dFa} \left(\text{C2}^2-\text{C3}\right)-6 a (a-1) \text{C2} \text{C3} \text{dFa}-4 (a-1)^3 \text{C4} \text{dFa}\right)$+\\$e^2 \left(2 a \text{C2}^2 \text{dFa}+3 (a-1)^2 \text{C3} \text{dFa}\right)-2 e ((a-1) \text{C2} \text{dFa})+O[e]^4$

\textbf{z = Simplify[e - ((b/dFy) + (c/dFx)) Fx]}\\
$e ($-$b$-$c$+$1)$+$e^3 \left(b \left(\left(-4 a^2+8 a-2\right) \text{C2}^2+\left(3 a^2-6 a+2\right) \text{C3}\right)+2 c \left(\text{C3}-\text{C2}^2\right)\right)$+\\$e^4 (b \left(\left(-8 a^3+28 a^2-26 a+4\right) \text{C2}^3+\left(12 a^3-39 a^2+38 a-7\right) \text{C2} \text{C3}+\left(-4 a^3+12 a^2-12 a+3\right) \text{C4}\right)$+\\$c \left(4 \text{C2}^3-7 \text{C2} \text{C3}+3 \text{C4}\right))+\text{C2} e^2 (-2 a b+b+c)+[e]^5$

\textbf{Solve[\{(1 - b - c) == 0, (b - 2 a b + c) == 0\}, \{a, b, c\}]}\\
$\left\{\left\{a\to \frac{1}{2 b},c\to 1-b\right\}\right\}$\\\\
Donde observamos que los valores de los parámetros obtenidos se corresponden con los presentados en la demostración analítica del orden de dicho método (Sección \ref{ordenhg}).